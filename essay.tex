\documentclass[11pt, oneside]{article}   	% use "amsart" instead of "article" for AMSLaTeX format
%\usepackage{geometry}                		% See geometry.pdf to learn the layout options. There are lots.
%\geometry{letterpaper}                   		% ... or a4paper or a5paper or ... 
%\geometry{landscape}                		% Activate for rotated page geometry
%\usepackage[parfill]{parskip}    		% Activate to begin paragraphs with an empty line rather than an indent

\usepackage{geometry}
 \geometry{
 a4paper,
 total={170mm,257mm},
 left=20mm,
 top=20mm,
 }

\usepackage{graphicx}				% Use pdf, png, jpg, or eps§ with pdflatex; use eps in DVI mode
								% TeX will automatically convert eps --> pdf in pdflatex		
\usepackage{amssymb}
\usepackage{amsmath}
\usepackage{fancyhdr}
\usepackage[utf8]{inputenc}
\usepackage[english]{babel}
\usepackage{enumerate}
\usepackage{arcs}
\usepackage{cancel}
\usepackage{xfrac}
\usepackage{soul}
\usepackage{tikz}

%SetFonts

%SetFonts

\usepackage[inline]{asymptote}


\pagestyle{fancy}
\fancyhf{}
%\rhead{Teacher David @ 18601688612}
\lhead{\leftmark}


\title{Writing Drafts}
\author{Cindy Hu}
%\date{}							% Activate to display a given date or no date

\begin{document}
\maketitle




\section{Sensoji Temple}

Sneaking into the shadows of crowds, I find there is no space. The Sensoji street demonstrates its prosperity under the flaming sky. From all directions, people are standing under the spotlight, holding hands with those closest to them. Drearily, it’s not so long until the heat melts me into the crowd, flowing down to the center of the street with those I’m not a part of. 

The Sensoji temple sits right at the end of the street, hiding behind the shadow of the clouds. But, from where I’m standing, shops fill up the sides of the pedestrian mall. People are pushing me forward. In the milky way of crowds that I’m in, various people are heading to the front.  

The air is humid in the crowds, and although the cool wind blows by, beads of sweat still drip off blushed cheeks. Suffering, adrift with the flow into the street center. While I’m unbuckling the buttons, trying to get some cool wind in, I find myself in a circle of maidens, who are dressed in stunningly ornate clothes and adorned with fabric jewels. Nevertheless, the maiden’s high-pitched giggle and teasing of each other have drowned out any other sounds. Whenever they stop, there is always a higher pitch sound rushing out of their mouth and getting louder and higher and louder and higher and louder and higher.  
  
Finally, cast off those maidens at a crossroads, I can feel the autumn wind again. Standing away from the street center, nearby a store, a young couple sits hand-in-hand. As a flicker of hair hangs in front of her face, the man gently caresses it back behind her ear. He softly strokes her cheek and moves his mouth to say something. It is surely something beautiful because a bashful grin appears on the young woman’s face. She unpacks her little bag and reaches her phone high up in the sky. The couple both laugh radiantly at the camera, recording this special moment. 

Wandering in the square with the crowds, now the side of the street seems apparent. Moving past the shops, away from the central street, the traditional buildings remind me of the history hundreds of years ago. Dreamily, reaching out my arms, enjoying the blow of chilly wind in the early autumn. Trees are flamed, reflecting the fire-like shadow on the store’s roofs and the temple, decorating the simple white walls with the comeliness of nature. The wind blows through the trees, knocking down two or three cherry-colored leaves. I stand below the trees and see a man walking purposely yet carefully down to the temple. He has a black bun on top of his head, pointing straight to the sky. Seeing a fan slightly waving in his hands, my sight follows him into the temple gate.   

Suddenly, someone bumps into me and breaks my illusion. A man in a shirt rushing by, he is still on his phone while he dashes up the street. Papers soar out of his briefcase and get caught by the wind, traveling to an unknown place on this street. Passers pass by, not giving a glance at the young adult. 

He stands dumbfounded, the phone is still ringing, indistinct voices can be heard: “…what’re…you thinking…you…here, now.” The group of maidens meets again at this crossroad, teasing and giggling again to what they have seen or done. They ignore the man, too. The man stoops down trembling and picks up the papers that are in his sight. His shirt is translucent because of the cold sweat, I can nearly see his slender limbs shiver. Zipping up my coat, turning my head the other way, I move directly towards the temple.  

At the end of the street, sits the fedora-like building. The Sensoji Temple reaps my sight. From the outside, white, red, and black intricate patterns complete the austere gate-door. The crowd bottleneck and thrust through the austere gates. They never raise their heads to admire the flowery fractal upon them. It’s only a line of black hair before the gate that others can see. 

Under the late evening light, scarlet pillars with dark bricks become iridescent. A gigantic lantern hangs below the ceiling of the major door, slightly swinging on the hanger, giving out weak yellow light. As the bottleneck is released, the crowd separates. Near the temples, but not in them, people are cleaning their hands at the Mitarai. Clear cool liquid flows down from the hishaku—a roughened ladle—and drips down, delicately. The droplets of water sparkle when cascading, taking away any impurities. 
  
Turning around. A dark blue line replaces the rosy line on the horizon. The crowded river slowly fades away and the modern streets appear. In this modern city, skyscrapers stand like bounding walls around the temple. 
  
My heart bumps.  

\newpage
\section{A Christmas Carol}
\subsection{How does Charles Dickens present the character of Scrooge in A Christmas Carol? }

In the novella ``A Christmas Carol'', Charles Dickens presents five stages of Scrooge. The beginning stage is about Scrooge as a critical and unsociable miser; the next three stages are about the evolution of Scrooge's opinion towards Christmas and how he treats others; the last stage is about the newborn Scrooge who feels happiness almost every moment. In this essay, I compare the different characteristics of Scrooge, along with the literacy devices Dickens used, before and after Scrooge's rebirth. As a reminder, the word \emph{stave} here means ``chapter''. 

Charles Dickens first foreshadows the changes in Scrooge in the title. Christmas in Christianity represents the birth of Jesus. Each year, Christians will gather to celebrate and show their respect to Jesus. The purpose of the setting at Christmas is to highlight the rebirth of Scrooge after Christmas Eve. ``The Spirits have done it all in one night,'' paraphrases the fact in the book that Christmas Eve was Scrooge’s rebirth and Christmas Day was his first day after rebirth. In stave five, Scrooge was described as ``quite a baby'', referring to the birth of him. The reborn Scrooge always carries a mood of pure joy, making him a character totally different from the past. ``Baby'' refers to the fact that Scrooge saw things in optimistic and pure ways after he changed his view on relationship with others, foreshadowing the start of his new life. 

First, I will expand Scrooge's personality before his rebirth. Early in stave one, Dickens described Scrooge mainly as a materialistic and disconnected old man. In the background of the Victorian age's Christmas, Scrooge's action of not celebrating Christmas is surely a strange action in British culture. This shows that he is disconnected from his surroundings. At the beginning of the novella, Dickens described the simile of Scrooge as ``hard and sharp as flint'',  which meant that he was a cold and harsh person. The word ``flint'' stands for a kind of rock that can light up a fire. ``Flint'' is used here with ``hard and sharp'' can insinuate a fact that Scrooge refused to help others, connecting to the fact that he is unsociable. Evidenced by the fact that he acted ``indignantly'' when his nephew came to invite him to the Christmas family dinner. 


Another characteristic of Scrooge in the early stages is harshness. Referring to the background information, this is the sneer towards the attitude of rich people to the poor law, workhouses, and prisons. By using a pathetic fallacy, Dickens created a ``cold, bleak, biting weather: foggy withal'' Christmas Eve, and therefore reflect the characteristic of harshness on Scrooge. The adjectives ``cold, bleak, biting'' is also suitable to describe Scrooge because of his cold-blooded nature, evidenced by his refusal to donate to charity and scaring away the singer that's blessing him. Dickens made a comparison between the normal Christmas Eve and Scrooge's Christmas Eve to show his mean life. In the novella, when Dickens was describing Scrooge's Christmas Eve, he used adjectives that have negative emotions which mainly created a depressed and chilly atmosphere to describe the weather. Also, Scrooge did not have expectations for Christmas: he took his ``melancholy dinner in his usual melancholy tavern'' and walked back to his ``dreary'' house. However, when Dickens was describing other people in this village, he described that some boys were sliding through streets and people on street were enjoying Christmas. Scrooge's keeping out from society strongly suggests that he has an insufferable personality.  As a result, Scrooge as a cynical person at this stage didn't have good relationship with people around.


In stave two to four, this is when Scrooge started to change his mind towards Christmas and society. In stave two, the past ghost showed readers Scrooge’s childhood. At this stage, features like ragged schools, boarding schools, and workhouses are clear in the setting of the story, presenting the differences between the poor and rich. Through what the ghost said to Scrooge: “a solitary child, neglected by his friends, is left there still”, we can see the reason why Scrooge has grown up to be a disconnected person: not being given a chance to touch the light of passion, and the fact that not knowing how to connect correctly with others. In stave 2, Scrooge was being touched by the images of his happy memories in the past Christmas, we can not only see his tragic pasts, but also the inside of his hard shell. Scrooge has a sealing attitude and did not know how to treat people correctly. Dickens built Scrooge as a pessimistic person, his neglected childhood, and eagerness for money enriched the reasons for him to grow up into a critical and unsociable miser. In this stave, as an introduction, Dickens built up a character that is close to some rich in the Victorian Age. Through this stave, his readers begin to understand Scrooge, including the background that supports the appearance of Scrooge as an old miser. 


In stave four, Scrooge as a rich person was being resisted by the poor people after his death, referring to the fact that in the Victorian Age poor people did not like rich people due to their ignorance and want. As I mentioned in the last paragraph, Scrooge was being built as a typical rich in the Victorian Age. This stave is when Scrooge faced his death in the future which made him fear his original ending---die lonely. Dickens explored Scrooge’s personality through some minor characters, such as the poor woman that grabbed his things after he died for better use and the couple that was happy because of his death because they no longer needed to pay for such a high rent. Take the woman as an example, “a little heavier judgment” was her want towards Scrooge’s death. To a personal prospect, it is an attitude that poor people have towards rich people. Through the relationships between poor and rich people in stave four, Dickens had successfully highlighted the negative relationship between poor people and rich people. As what was being mentioned in stave three, there was a relationship of “ignorance and wants” between classes: rich people ignoring what poor people want. By characterizing Scrooge in a classic Victorian Age rich person, it’s easy for Dickens to criticize and punish him---rich people---using what future ghosts presented to Scrooge and Scrooge’s fear towards the bad end. By this, he can create the theme of this novella: criticizing rich people’s guilty personalities and their ignorance towards the classes that need help. 


As a conclusion for Scrooge before rebirth, Scrooge was a representation of rich people in the Victorian Age. What Dickens believed is that the people that can help others but ignore those who need help were sinners, at this point, the rich people in Victorian Ages. Dickens used Scrooge’s characteristics and his experiences with ghosts to discuss and criticize rich people. He criticized the rich people’s materialistic and unsociable by using the fact that Scrooge did not have anyone close to him and the fact that Scrooge died lonely and had the bad end that has a significantly huge difference between what he should get due to his great fortune and what he got, showing the central theme which is criticizing rich people’ relentless. However, what Dickens planned for the character Scrooge also tells the readers that sinners can confess and make atonement for the poor people to change their bad end. I would argue that Dickens criticized rich people because of his sympathy but suggested ways to atone because he believed in religion and their power to give poor people a good life. 


At this rebirth stage, Scrooge, as a penitent, is looking for forgiveness and remedy. Dickens had presented Scrooge as a person that’s afraid of his original future and starting willing to change his future. 


In stave five, Scrooge is no more the miser but a newborn passionate man. In this chapter, Dickens used the literary technique pathetic fallacy differently than chapter one, “heavenly sky” is what Dickens used to describe Christmas day. Compared to the “heaviest rain” on Christmas Eve in chapter one, the weather performed Scrooge’s happiness on Christmas Day after his re-birth and after he changed his attitude towards social lives. Dickens presents Scrooge from an unsociable, materialistic old man into a happy and extrovert “schoolboy”. The phrase “happy as an angel” is how Dickens described Scrooge in chapter 5. “Angel” here can be understood as the person being realized from chains, referring to the “chains” that were surrounding him in chapter one. Also, referring to Christianity, “angels” were people that do not have sins and dedicated a lot to others. The phrase states Scrooge as a vivid and energetic person after his changes and the fact that he lived positively and enjoy a good life. 


Dickens used Scrooge as the classic rich person in the Victorian age to criticize the ignorance from rich people and show his sympathy towards weaker communities. Religion is a big part that flows between chapters in this novella, features like candles, fire, and church have built up the background of this story and gave reasons to some fantasy creatures and plots in the novella. Dickens had successfully presented Scrooge’s changes from a ‘negative’ person to a ‘positive’ person, which suggests a new style of relationship between poor people and rich people that’s not only “ignorance and want”. 





\end{document} 